\section{Future Work}
SuperGo is able to beat most of its opponents in the small competition. But still a lot of work remains to be done. We think the following tips can improve SuperGo's performance or deserve further investigation.

\begin{itemize}
\item Dynamic Komi. According to our observation, dynamic Komi enhanced SuperGo's performance against its performance. And this technique is actually applicable to all MC algorithms--dynamically changing the criteria of winning/losing to make Monte-Carlo simulation provide more discriminating results. We plan to test this technique in Fuego and see how it works.

\item Pattern Matching. Compared to other advanced Go agents such as GNUGo, pattern matchings in SuperGo is far from accurate. In GNUGo, every pattern is annotated with additional information and constraints. Patterns in GNUGo are less rigid but can be used more appropriately by machines. 

The current problem of pattern matching in SuperGo is that moves generated by pattern matching tend to be densely located in a small region. This is not a optimum strategy in the opening phase of the game.
In future, more abundant and clever patterns will be implemented in SuperGo.

\item More Heuristics. Currently, SuperGo has not implemented all the heuristics. We may try to create more eyes, connect existing chains and prevent the opponents from connecting their chains.

\item Safety Solver. We may use more sophisticated algorithms like the unconditional life algorithm to prune the unnecessary moves in Monte-Carlo simulations. In this way we can make the simulations more efficient and provide more accurate predictions. 

\end{itemize}