\section{Construction and Usage of Opening Books}

Games are usually divided into three phases: opening, middlegame and endgame. In any of the three phases, the default action of a computer program is to start a brute force search for a ``best'' move. Since such a search has to be performed within a limited time, it can only examine nodes down to a certain depth, which means that the calculated value is only a heuristic approximation of the game-theoretic value.

Under such circumstances, strategies of using precalculated ``references'' have come into being. Such ``references'', in the parlance of games, are called books. Books specially used in opening phases of the game are called opening books, while books used in endgames are endgame books.


\subsection{Opening Books for Go}

An opening book is an important feature of any game-playing computer program. These books used to be constructed manually by an expert, by storing good moves suggested by theory, or simply by listing all games ever played by
strong players.

An opening book is especially useful in Go games since Go games have a extremely large search space in the opening phase of the game. Using a precalculated book will greatly improve the performance of a computer go and save the time in a time-limited competition.

During the history of Go game, human players have generalized several systematic theories in the opening phase of Go game. Fuseki, a Chinese word coined to represent initial sequence of moves in a game, is famous among professional go players. Only a proportion of fusekis have recognised or specific names. These include the two-star fuseki, three-star fuseki, Chinese fuseki, Kobayashi fuseki, and Shusaku fuseki.

Generally, there are two types of fusekis, both of which have their own theoretical background:
\begin{itemize}
\item \textbf{Territorial approach}
As played on a large board (the standard 19x19 line goban), the priority is given to playing corner enclosures, then to extending to the middle of the sides, and finally to the center because it is easier to secure territory in the corners than on the sides or in the center.
\item \textbf{Influence-oriented approach}
Unlike the territory-oriented playing style, this approach emphasizes control of the center. The reason for this is that one's play should not be narrowly focused on attempting to secure points quickly by occupying the corners first. Although it requires more effort to secure the center, it is more profitable since the center constitutes the majority of territory on the board.
\end{itemize}

\subsection{Opening Books in SuperGo}

In our SuperGo project, we adapt the opening book from Feugo. As the opening book of Feugo is largely for the 19x19 line goban, we have to process it in order to make it usable and reasonable in the 13x13 line goban. We use a mapping which maintains the locations of 9 special points on the goban. To be more specific, a rough mapping table is provided:

\begin{center}
\begin{tabular}{c|c}

19x19 & 13x13 \\ \hline
D4 & D4 \\
D10 & D7 \\
D16 & D10 \\
K4 & G4 \\
K10 & G7 \\
K16 & G10 \\
Q4 & K4 \\
Q10 & K7 \\
Q16 & K10 \\
\end{tabular}
\end{center}

The other points in the 19x19 line goban is randomly mapped into the 13x13 line goban according to the adjacent situations of the point on the board.

Since the board in the opening phase of the game contains very few stones, it is rather cumbersome to represent the whole board in the opening book. Instead, we use a sequence of strings to represent a sequence of initial moves of a game. For example,
\begin{verbatim}
Q4 Q16 C4 D16 C14 F17 | D10
\end{verbatim}
represents the first six moves followed by a next move D10 suggested by the book. This representation is cheap to store and easy to match.
