\section{Pattern Matching}

This part integrates some of knowledge of human go players into SuperGo. We have stored some existed patterns (or Joseki) in our SuperGo to make it ``think'' more like a human player. 

When several candidate moves have been generated by Monte Carlo simulation, each move is then matched to see if it follows a Joseki stored in our pattern database. If a candidate move matches a Joseki quite perfectly, it is highly likely that this is a quite good move. Therefore, a candidate move that matches a Joseki is given higher score so that a move generator will have a higher possibility to choose it.

Compared with Monte Carlo simulation, pattern matching is more rigid and yields high efficiency. Actually, using pattern matching techniques in go game is similar to ``standing on the giants' shoulders''.

\subsection{Pattern Representation}

Most of the patterns we use in SuperGo only consider the local situations. Hence, we use a standard way to represent these patterns. We use two-dimensional char arrays to store local patterns. The examples are as follows:

\begin{verbatim}
                               ?.??           ??xx?
                               O*.O           O.xx?
                               .X..           .?X.O
                                              .O*..
                                              .....
                                              -----
\end{verbatim}

The different symbols stand for different meanings, and they are shown as follows. Such representations are easy for the human to read and comprehend.

\begin{center}
\begin{tabular}{l|l}
\textbf{Symbols} & \textbf{Meanings} \\ \hline
? & don't care \\
. & empty \\
X & your stone \\
O & your opponent's stone \\
x & your stone or empty \\
o & your opponent's stone or empty \\
* & your next recommended move \\
$-$ or $\mid$ & edges of board \\
$+$ & corner of board \\
\end{tabular}
\end{center}

\subsection{Matching and Scoring}

We have stored over 2000 patterns in our pattern database. When doing matching, we have also taken into consideration rotations, reflections of patterns. Therefore, we have a combination of over 100000 patterns in SuperGo.

Candidate moves that satisfy more patterns will receive a higher score, matching a corner pattern will earn a candidate move higher score, and matching a larger pattern (8X8 array) will be given higher rate.

However, pattern matching in SuperGo is far from accurate compared to GNUGo. In GNUGo, every pattern is annotated with additional information and constraints. Patterns in GNUGo are less rigid but can be used more appropriate by machines. In future, more abundant and clever patterns will be implemented in SuperGo. 